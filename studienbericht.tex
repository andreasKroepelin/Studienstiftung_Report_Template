% Erstellt nach der Vorlage für Microsoft Word, abgerufen aus dem Daidalosnet in der Version vom 15.11.17

\documentclass[]{scrartcl}
\usepackage[utf8]{inputenc}
\usepackage[T1]{fontenc}
\usepackage[ngerman]{babel}
\usepackage[]{graphicx}
\usepackage[bottom=3cm, top=3cm]{geometry}

\titlehead{\includegraphics[height=10ex]{img/stusti_logo.jpg}}
% Jahr ergänzen, alternativ auch Semester statt Studienjahr
\title{Studienbericht über das zurückliegende Studienjahr}
\date{}

\begin{document}
\maketitle

\begin{tabular}{p{.5\textwidth}p{.5\textwidth}}
    Name, Vorname: & \\
    E-Mail: & \texttt{} \\
    Hochschule: & \\
    Studienfach: & \\
    Aktuelle Zahl der Fachsemester: & \\
    Aktuelle Semesterzahl: & \\ % Studiensemester seit Studienbeginn
    Angestrebter Studienabschluss in der aktuellen Studienphase: & \\ % z. B. Bachelor, Master, Staatsexamen
    Vertrauensdozent/-in: & \\
    Referent/-in: & \\
\end{tabular}

% Der ausführliche Studienbericht ermöglicht Vertrauensdozent/-in und Referent/-in Einblick in Ihren Studienverlauf, Ihre Aktivitäten und Interessen außerhalb des Studiums und Ihre weiteren Pläne.
% Gleichzeitig soll der Bericht Sie zu einer persönlichen Reflexion über das zurückliegende Semester anregen.
% Falls Sie Fragen oder Gesprächsbedarf haben oder sich wesentliche Änderungen außerhalb dieses Berichtsturnus ergeben, insbesondere im Bereich Ihrer Studienplanung, wenden Sie sich bitte jederzeit direkt an Ihre Ansprechpartner/-innen in der Geschäftsstelle.

\section*{Einleitende Zusammenfassung}

\paragraph{Auf diesem Stand ist jetzt mein Studium:}

\paragraph{Wichtige Studienbezogene Änderungen:}
% Meine studienbezogene oder persönliche Situation wird sich verändern oder hat sich bereits wesentlich verändert (im Bereich des Studiums z.B. Wechsel des Studienfachs, Wechsel der Hochschule, Auslandsaufenthalt, Wechsel in der Ausrichtung des Studienziels, Unterbrechung des Studiums oder verzögerter Studienverlauf, längere nicht-obligatorische Praktika; im persönlichen Bereich z.B. Erkrankung, Eheschließung, Geburt eines Kindes).
% -> Nein/Ja      
% (Änderungen bitte hier kurz nennen und ggf. im Bericht näher erläutern)
%
% Hinweis zu finanziellen Änderungen:
% Sollten Sie ein Grundstipendium erhalten und Ihre finanzielle Situation ändert sich (z.B. Änderung Ihrer Einkünfte, Änderung der finanziellen Situation Ihrer Familie), teilen Sie diese Änderung bitte direkt dem für Sie zuständigen Ansprechpartner in unserer Stipendienberechnung mit.

\paragraph{Das war für mich außerhalb des Studiums von großer Bedeutung:}

\paragraph{Für das nächste Studienjahr/Semester habe ich folgende Pläne:}

\section*{Ausführlicher Bericht}
% 2 bis 3 Seiten

\end{document}
